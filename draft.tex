\documentclass[11pt]{article}

\usepackage{amsmath}

\begin{document}

\section{Recap}

Functor laws:

\begin{align}
    fmap ~ id =& ~ id
    \label{1st-functor-law} \\
    fmap ~ (f \cdot g) =& ~ fmap ~ f \cdot fmap ~ g
    \label{2nd-functor-law}
\end{align}

\section{Fold typeclass}

Laws

\begin{align}{lr}
    toRecF \cdot fromRecF =& ~ id
    \label{to-rec-f-id} \\
    fromRecF \cdot toRecF =& ~ id
    \label{from-rec-f-id}
\end{align}

Note: the following proof only holds for \emph{finite} data structures, where we
have a base case.

\begin{equation}
  fold ~ fromRecF = id
\end{equation}

Base case for lists:

\begin{equation*}
  \begin{array}{lll}
     & fold ~ fromRecF ~ [] \\
    =& fromRecF ~ (fmap ~ (fold ~ fromRecF) ~ (toRecF ~ []))
     & \text{(definition fold)} \\
    =& fromRecF ~ (fmap ~ (fold ~ fromRecF) ~ NilF)
     & \text{(definition toRecF)} \\
    =& fromRecF ~ NilF
     & \text{(ListF functor)} \\
    =& []
     & \text{(definition fromRecF)} \\
  \end{array}
\end{equation*}

Can we make this more abstract by relying on structure from RecF?

Induction step:

\begin{equation*}
  \begin{array}{lll}
     & fold ~ fromRecF \\
    =& fromRecF \cdot fmap ~ (fold ~ fromRecF) \cdot toRecF
     & \text{(definition fold)} \\
    =& fromRecF \cdot fmap ~ id \cdot toRecF
     & \text{(induction hypothesis)} \\
    =& fromRecF \cdot toRecF
     & \text{(see \ref{1st-functor-law})} \\
    =& id
     & \text{(see \ref{from-rec-f-id})} \\
  \end{array}
\end{equation*}

\end{document}
